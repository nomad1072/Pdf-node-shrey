
        \documentclass{article}
        \usepackage[left=1.5cm, right=5cm, top=2cm]{geometry}
        \usepackage[utf8]{inputenc}
        \usepackage{longtable,pdflscape,graphicx}
        \begin{document}
        \pagenumbering{gobble}
        
                \newpage
                \section{Section F1: Product Attributes Table}         
                The product attributes are given below for Product Type: solid
                \begin{longtable}{ |p{1.5cm} |p{1.7cm} |p{1cm} |p{1.5cm} |p{1.7cm} |p{1cm} |p{1cm} |p{1cm} |p{1cm} |p{1.5cm} |}
                \hline

                Name & Product Id & API & Solubility Factor & Cleanability Factor & PDE & Min TD & Max TD & Min BS & Strength\\

                \hline

    Product1 & P1 & API1 & 1 & 1 & 0 & 180 & 360 & 440000 & 400\\
\hline
Product2 & P2 & API2 & 3 & 3 & 0 & 180 & 360 & 210000 & 400\\
\hline
Product3 & P3 & API3 & 5 & 3 & 0 & 300 & 600 & 200000 & 400\\
\hline
Product4 & P4 & API4 & 2 & 5 & 1 & 150 & 300 & 200000 & 400\\
\hline

    \end{longtable}
    
        
        \newpage
        \section{Section F2: Equipment Attributes Table}
        The equipment attributes are given below:
        \begin{longtable}[l]{ |p{3cm} |p{3cm} |p{2cm}|}
        \hline

        Equipment Id & Equipment Name & Surface Area\\

        \hline

    EQ1 & Equipment1 & 10000\\
\hline
EQ2 & Equipment2 & 20000\\
\hline
EQ3 & Equipment3 & 40000\\
\hline

    \end{longtable}
    
        
        \newpage
        \section{Section F3: Equipment Group Attributes}
        The Equipment Group attributes are given below:
        \begin{longtable}[l]{ |p{2cm} |p{2cm} |p{2.5cm} |p{6cm}|}
        \hline

        Name & Group Id & Product Type & Equipments\\

        \hline

    Train1 & EqGrp1 & solid & Equipment1, Equipment2, Equipment3\\
\hline

    \end{longtable}
    
        
        \newpage
        \section{Section F4: PE Matrix}
        The PE (Product-Equipment) relationship is described by the table given below:
        \begin{longtable}[l]{ |p{3cm} |p{5cm} |p{3cm}|}
        \hline

        Product Id & Equipment Used & Surface Area\\

        \hline

    P1 & EQ1, EQ2, EQ3 & 70000\\
\hline
P2 & EQ1, EQ2 & 30000\\
\hline
P3 & EQ1, EQ3 & 50000\\
\hline
P4 & EQ1, EQ2, EQ3 & 70000\\
\hline

    \end{longtable}
    The PE (Product-Equipment) relationship is described by the table given below:
    \begin{longtable}[l]{ |p{3cm} |p{5cm} |p{3cm}|}
    \hline

    Product Id & Equipment Used & Surface Area\\

    \hline

    Equipment1 & P1, P2, P3, P4 & 10000\\
\hline
Equipment2 & P1, P2, P4 & 20000\\
\hline
Equipment3 & P1, P3, P4 & 40000\\
\hline

    \end{longtable}
    
        
        \newpage
        \section{Section F5: Calculation Variables}
        The various variables used in the evaluation of the worst case limits and molecules are given in the table given below:
        \begin{longtable}[l]{ |p{5cm} |p{2cm} |p{1.5cm} |p{6cm} |p{2.5cm}|  }
        \hline

        Name & Short Name & Unit & Description & Default Value\\

        \hline

    Body weight (for LD50 dose) & bw & kg & body weight of patient taking next product & 60\\
\hline
Equipment Verification Period & verifyPeriod & month & The frequency at which equipment needs to be verified. & 3\\
\hline
Modification factor & mf &  & Cumulative modifying factor, selected by the toxicologist. generally no more than 1000 & 1000\\
\hline
Safety Factor - Solids & sf\_solid &  & Safety Factor for Solids drug dosage & 1000\\
\hline

    \end{longtable}
    
        
    \newpage
    \section{Section F6: MAC Formula}
    The below set of formula are used to calculate the MAC Limits. Please note that the MAC Surface Area is taken as the default limit here.
    \begin{longtable}[l]{ |p{2cm} |p{1.5cm} |p{1.5cm} |p{5cm} |p{6cm}|  }
    \hline

    Name & Sampling Type & Product Type & Formula & Description\\

    \hline

    MAC\_dosage & swab & solid & (1 / sf\_solid) * (min\_td\_a) * (1 / max\_td\_b) * (min\_bs\_b) * (1 / area\_shared) & based on minimum daily dose of the drug active in a maximum daily dose of the next drug product\\
\hline
MAC\_general & swab & solid & (1 / 1e+5) * (min\_bs\_b) * (1 / area\_shared) & general 10ppm limit to be considered when it is lower than dosage/toxicity based limits or when dosage/toxicity data is not available\\
\hline
MAC\_toxicity & swab & solid & (pde\_a) * (1 / max\_td\_b) * (min\_bs\_b) * (1 / area\_shared) & based on Risk-MaPP Acceptable Daily Exposure (ADE) approach\\
\hline

    \end{longtable}
    
        
                \newpage
                \section{Section F7: Sampling Paramters}         
                The Sampling Parameters used in the protocol workflow is as:
                \begin{longtable}[l]{ |p{3.6cm} |p{1cm} |}
                \hline

                Sampling Parameter & value\\

                \hline

    swab & 40\\
\hline
sda & 20\\
\hline

    \end{longtable}
    
        
                \newpage
                \section{Section G2: Risk Formula}         
                Risk Priority Number(s) are defined as per the formula given in the table below:
                \begin{longtable}[l]{ |p{2cm} |p{8cm} |p{1.2cm} |p{3cm} |}
                \hline

                Name & Description & Rank & Formula\\

                \hline

    RPN\_overall & risk evaluation from multiple risk factors & 1 & R1*R2*R3*R4\\
\hline

    \end{longtable}
    
        
                \newpage
                \section{Section H: Current Cleaning Limit Policy}         
                Current Cleaning Limit Policy given in the table below:
                \begin{longtable}[l]{ |p{3.5cm} |p{14cm} |}
                \hline

                Name & Description\\

                \hline

    Default & Recommended cleaning limit policy based on latest regulatory   guideline. Acceptance limit is always equal to HBEL based limit and site acceptance limit is either based   on dosage based limit if it significantly lower than HBEL or is a lower ratio of HBEL itself\\
\hline

    \end{longtable}
    
        \end{document}
        